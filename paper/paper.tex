\documentclass{article}

\makeatletter
% Allow compiling either from the repo root (`pdflatex paper/paper.tex`) or
% from within `paper/` (`pdflatex paper.tex`).
\def\input@path{{./}{paper/}}
\makeatother

\newcommand{\R}{\mathbb{R}}
\newcommand{\N}{\mathbb{N}}
\newcommand{\Z}{\mathbb{Z}}
\newcommand{\E}{\mathbb{E}}
\newcommand{\matchal}[1]{\mathcal{#1}}
\newcommand{\A}{\mathcal{A}}
\newcommand{\C}{\mathcal{C}}
\newcommand{\NF}{\matchal{N(F)}}
\newcommand{\NFA}[1]{\matchal{N(#1)}}
\newcommand{\EF}{\mathcal{E}(\mathcal{F})}
\newcommand{\EFA}[1]{\mathcal{E}(\mathcal{#1})}
\newcommand{\RF}{\mathcal{R}(F)}
\newcommand{\RFA}[1]{\mathcal{R}(\mathcal{#1})}
\newcommand{\T}{\mathcal{T}}
\newcommand{\GF}{\mathcal{G}(F)}
\newcommand{\SGF}{\mathcal{SG}(F)}
\newcommand{\SGFA}[1]{\mathcal{SG}(#1)}
\newcommand{\GFV}[1]{\mathcal{G}(#1)}
\newcommand{\F}{\mathcal{F}}
\newcommand{\len}{\text{len}}
\newcommand{\D}{\mathcal{D}}
\newcommand{\argmin}[2]{\underset{#1}{\text{argmin }} {#2}}
\newcommand{\argmax}[2]{\underset{#1}{\text{argmax }} {#2}}
\newcommand{\M}{\mathcal{M}}
\newcommand{\prt}{\partial}
\usepackage{mathtools}
\usepackage{amsmath}
\usepackage{physics}
\newcommand{\G}{\mathcal{G}}
% If you need to pass options to natbib, use, e.g.:
%   \PassOptionsToPackage{numbers,compress}{natbib}
% before loading paper.

% ready for submission
\usepackage{neurips_2024}

% Standard packages
\usepackage[utf8]{inputenc} % allow utf-8 input
\usepackage[T1]{fontenc}    % use 8-bit T1 fonts
\usepackage{hyperref}       % hyperlinks
\usepackage{url}            % simple URL typesetting
\usepackage{booktabs}       % professional-quality tables
\usepackage{amsfonts}       % blackboard math symbols
\usepackage{nicefrac}       % compact symbols for 1/2, etc.
\usepackage{microtype}      % microtypography
\usepackage{xcolor}         % colors
\usepackage{graphicx}       % figures
\usepackage{enumitem}       % compact lists
\usepackage{multirow}       % tables
\usepackage{array}          % tables
\usepackage{siunitx}        % numbers in tables

% Local macros
\newcommand{\vectur}{\textsc{VecTur}}
\newcommand{\vecstur}{\textsc{VecSTur}}
\newcommand{\compgen}{\textsc{CompGen}}

\title{VecTur: Vector Turing Machines}

\author{%
  Ethan Hall \\
  \texttt{ethan.hall.phd@gmail.com} \\
}

\begin{document}
\maketitle

\begin{abstract}
We introduce \vectur{} (Vector Turing Machines), a differentiable architecture for learning functions by vectorizing the tape symbols, head index, and finite control of a classical Turing machine \citep{turing1936}.
\vectur{} is designed to support \emph{deep computation} by explicitly iterating a learned transition map while learning an input-conditioned halting schedule via a parameter \(\kappa\).
We evaluate \vectur{} as a drop-in \emph{computational block} inside a Llama-style decoder-only language model \citep{touvron2023llama2,dubey2024llama3}, contrasting it with attention \citep{vaswani2017attention}, LSTM-style recurrence \citep{hochreiter1997lstm}, and differentiable external-memory baselines \citep{graves2014ntm,graves2016dnc}.
On small-to-medium open benchmarks for reasoning and language (GSM8K \citep{cobbe2021gsm8k}, ARC \citep{clark2018arc}, HellaSwag \citep{zellers2019hellaswag}, WikiText-103 \citep{merity2016wikitext}), we find \vectur{} improves algorithmic generalization at fixed parameter budgets.
We additionally introduce \compgen{}---a synthetic program dataset stratified by \((T(n),S(n))\) complexity classes---and show \vectur{} adapts its effective compute by learning \(\kappa\).
Finally, we propose \vecstur{}, a stochastic extension that consumes random tape symbols and outperforms \vectur{} on randomized verification tasks (e.g., Freivalds-style matrix product verification \citep{freivalds1977}).
\end{abstract}

\section{Introduction}
Modern language models excel at pattern completion yet often struggle to reliably \emph{execute} long algorithmic computations, extrapolate beyond training lengths, or allocate variable compute per input.
Several lines of work attempt to address these limitations by embedding algorithmic structure into neural systems, including external-memory architectures \citep{graves2014ntm,graves2016dnc} and adaptive computation mechanisms \citep{graves2016act}.

We propose \vectur{}, a vectorized analogue of a classical Turing machine \citep{turing1936} whose tape, head index, and finite control are represented as continuous vectors and updated by a learned transition map.
Unlike attention-only computation \citep{vaswani2017attention}, \vectur{} explicitly iterates a state transition; unlike Neural Turing Machines \citep{graves2014ntm}, \vectur{} emphasizes sparse head indexing and a learned halting schedule parameterized by \(\kappa\), enabling adaptive depth.

We evaluate \vectur{} in a realistic regime by inserting it as a \emph{block} inside a Llama-style decoder-only macro architecture \citep{touvron2023llama2,dubey2024llama3}, replacing the standard attention+MLP block.
We compare against alternative blocks: (i) standard attention \citep{vaswani2017attention}, (ii) LSTM-style recurrence \citep{hochreiter1997lstm}, and (iii) differentiable external-memory controllers \citep{graves2014ntm,graves2016dnc}.
We focus on small-to-medium models (roughly \(10^8\) to \(10^9\) parameters) where architectural inductive bias can materially affect sample efficiency and extrapolation.

We further introduce \compgen{}, a dataset of generated Python programs grouped into discrete complexity buckets \((T(n),S(n))\) such as \(O(n)\)/\(O(1)\), \(O(n\log n)\)/\(O(1)\), \(O(n^2)\)/\(O(1)\), and \(O(n^2)\)/\(O(n)\).
The goal is to probe whether a model can learn to \emph{compute} across increasing \(n\) by allocating more steps as needed, rather than memorizing only small \(n\).
Finally, we define \vecstur{}, which augments the input with stochastic symbols \(z\) to emulate randomized computation, and we propose a randomized evaluation suite where randomness yields asymptotic speedups \citep{freivalds1977,miller1976,rabin1980}.

\paragraph{Contributions.}
\begin{itemize}[leftmargin=*,nosep]
  \item We define \vectur{}, a vectorized Turing-machine-inspired transition system with sparse indexing and a learnable halting schedule \(\kappa\).
  \item We propose a plug-and-play integration of \vectur{} as a computational block inside Llama-style decoder-only models.
  \item We introduce \compgen{}, a synthetic program dataset labeled by time/space complexity class \((T(n),S(n))\), and a protocol for extrapolation across \(n\).
  \item We define \vecstur{} and a randomized computation evaluation suite, showing benefits of stochastic symbols for problems with randomized speedups.
\end{itemize}

\section{Related Work}
\paragraph{Sequence models and attention.}
Transformers \citep{vaswani2017attention} and their decoder-only variants power modern LLMs (e.g., Llama-family models \citep{touvron2023llama2,dubey2024llama3}).
Recurrent networks such as LSTMs \citep{hochreiter1997lstm} provide a different inductive bias for iterative computation but historically underperform attention-based models at scale on language modeling.

\paragraph{Differentiable memory and neural machines.}
Neural Turing Machines (NTMs) \citep{graves2014ntm} and Differentiable Neural Computers (DNCs) \citep{graves2016dnc} integrate external memory with differentiable read/write heads.
Our work shares the goal of improving algorithmic behavior, but emphasizes (i) sparse indexing for efficiency and (ii) a learned halting schedule.

\paragraph{Adaptive computation.}
Adaptive Computation Time (ACT) \citep{graves2016act} learns when to stop iterating; our \(\kappa\)-parameterization provides a simple, input-conditioned control knob for the effective number of steps.

\paragraph{Randomized algorithms.}
Randomness can reduce expected runtime for verification and decision problems; canonical examples include Freivalds' randomized matrix product verification \citep{freivalds1977} and probabilistic primality testing \citep{miller1976,rabin1980}.
\vecstur{} is intended as a neural analogue that can exploit stochastic symbols during computation.

\section{VecTur: Vector Turing Machines}

\subsection{Vectorized machine state}
Given an input sequence \(x \in \R^{N \times d_x}\) (e.g., token embeddings), we define a \vectur{} block below. Note that for \vecstur{}, we additionally sample a sequence of stochastic symbols \(z \in \R^{N_z \times d_x}\) and set the tape length
\begin{equation}
N_T = N + N_z,
\end{equation}
so that each input symbol and each stochastic symbol can be addressed at least once. In our experiments we use \(N_z \approx N\) (so \(N_T \approx 2N\)).

We define the machine state at step \(t\) as a triple \((T_t,Q_t,I_t)\), where the tape \(T_t \in \R^{N_T \times d_T}\), the control state \(Q_t \in \R^{d_Q}\), and the head index
\[
I_t = (\boldsymbol{\theta}_t,\mathbf{w}_t) \in (S^1)^k \times \R^k
\]
are learned, differentiable quantities. (Here \(\boldsymbol{\theta}_t=(\theta_{t,1},\ldots,\theta_{t,k})\) parameterizes \(k\) head locations on the circle and \(\mathbf{w}_t=(w_{t,1},\ldots,w_{t,k})\) are the associated weights.)
We index tape positions by \(j\in\{0,1,\ldots,N_T-1\}\), and write \(T_t[j]\in\R^{d_T}\) for the \(j\)-th tape symbol.

The initial state is produced by learned maps \(\M_T,\M_Q,\M_I\) with parameters \(W\):

\begin{equation}
    T_0=\M_T(x;W), \quad Q_0=\M_Q(x;W), \quad I_0=\M_I(x;W),
\end{equation}

where $\M_T, \M_Q, \M_I$ can be any mapping using some parameters $W$.

\subsection{Sparse addressing (keeping I and J fixed)}
Define the following piecewise linear map \(E:S^1\to \R^{N_T}\) as

\begin{align*}
    % Define the segment index n
    n(\theta) &= \left\lfloor \frac{N_T \theta}{2\pi} \right\rfloor \\
    % Define the local interpolation parameter s
    s(\theta) &= \frac{N_T \theta}{2\pi} - \left\lfloor \frac{N_T \theta}{2\pi} \right\rfloor \\
    % The function E
    n^+(\theta) &= (n(\theta)+1)\bmod N_T \\
    E(\theta) &= (1 - s(\theta)) e_{n(\theta)} + s(\theta) e_{n^+(\theta)}
\end{align*}

We will write any \(I\in (S^1)^k\times\R^k\) as \(I=(\boldsymbol{\theta},\mathbf{w})\), and define the induced sparse tape-index weighting vector \(J(I)\in \R^{N_T}\) by

\begin{equation}
    J(I) = \sum_{i=1}^k w_i\,E(\theta_i).
\end{equation}

By construction, each \(E(\theta_i)\) is supported on at most two adjacent tape locations \(\{n(\theta_i),n^+(\theta_i)\}\), hence \(J(I)\) is supported on at most \(2k\) tape locations. Concretely, for each head atom \((\theta_i,w_i)\) define
\[
n_i := n(\theta_i),\qquad s_i := s(\theta_i),\qquad n_i^+ := (n_i + 1)\bmod N_T,
\]
so that \(E(\theta_i)=(1-s_i)e_{n_i}+s_ie_{n_i^+}\). This gives an implementation-friendly form: one can store \((n_i,n_i^+, (1-s_i)w_i, s_i w_i)\) for each \(i\) and never materialize the dense \(N_T\)-vector \(J(I)\).

\subsection{Read, transition, and halting}
We define the transition map \(\Delta\) that updates the tape, control state, and head index. First, we use the head index \(J(I_t)\) to read a single tape symbol \(S_t\in\R^{d_T}\):

\begin{equation}
    S_t = \sum_{j=0}^{N_T-1} \big(J(I_t)\big)_j\,T_t[j] \in \R^{d_T}.
\end{equation}
Equivalently, using the explicit \(2k\)-sparse form above,
\[
S_t
= \sum_{i=1}^k w_{t,i}\Big((1-s_{t,i})\,T_t[n_{t,i}] + s_{t,i}\,T_t[n_{t,i}^+]\Big),
\]
so \(S_t\) is computed using at most \(2k\) gathered tape vectors, and is piecewise linear in the tape (and linear in the interpolation weights away from the measure-zero segment boundaries induced by the floor operation).


Next, define a gate \(g_t\in(0,1)\) that controls the effective amount of computation and enables early stopping. We use a sigmoid gate,

\begin{equation}
    g_t
    = \sigma\!\left(\frac{\kappa(x;W)-t}{\max\!\big(1,\norm{Q_t-q_0}^2\big)}\right),
\end{equation}
where \(\sigma(u)=1/(1+e^{-u})\), \(\kappa(x;W)>0\) is a learned scalar per example, and \(q_0\in\R^{d_Q}\) is a learned halting target state. Intuitively, larger \(\kappa(x;W)\) yields more effective steps (slower decay in \(t\)), while \(\|Q_t-q_0\|\) encourages the dynamics to become stationary near the target.

We update the tape, control state, and head index using learned transition maps \(\Delta_T,\Delta_Q,\Delta_\theta,\Delta_w\). Let
\[
U_t := \Delta_T(S_t,Q_t;W)\in\R^{d_T}.
\]
Then the update equations are

\begin{align}
    T_{t+1}[j] &= T_t[j] + g_t\,\big(J(I_t)\big)_j\,U_t \qquad \text{for } j\in\{0,\ldots,N_T-1\}, \label{eq:vectur_tape_update}\\
    Q_{t+1} &= Q_t + g_t\,\Delta_Q(S_t,Q_t;W),\\
    \boldsymbol{\theta}_{t+1} &= \big(\boldsymbol{\theta}_t + g_t\,\Delta_\theta(S_t,Q_t,\boldsymbol{\theta}_t;W)\big)\bmod 2\pi,\\
    \mathbf{w}_{t+1} &= \mathbf{w}_t + g_t\,\Delta_w(S_t,Q_t,\mathbf{w}_t;W),\\
    I_{t+1} &= (\boldsymbol{\theta}_{t+1},\mathbf{w}_{t+1}).\label{eq:vectur_head_update}
\end{align}

Equation~\eqref{eq:vectur_tape_update} makes the sparsity explicit: since \(\big(J(I_t)\big)_j=0\) for all but at most \(2k\) locations, only \(O(2k)\) tape vectors are updated per step. In an efficient implementation, \eqref{eq:vectur_tape_update} is executed as a scatter-add into those \(2k\) indices (and \(S_t\) is computed as a gather + weighted sum).

The transition maps have the following types:
\begin{align*}
\Delta_T &:\R^{d_T}\times\R^{d_Q}\to\R^{d_T},\\
\Delta_Q &:\R^{d_T}\times\R^{d_Q}\to\R^{d_Q},\\
\Delta_\theta &:\R^{d_T}\times\R^{d_Q}\times(S^1)^k\to\R^k,\\
\Delta_w &:\R^{d_T}\times\R^{d_Q}\times\R^k\to\R^k.
\end{align*}
The \(\bmod 2\pi\) in \eqref{eq:vectur_head_update} ensures the head angles represent elements of \(S^1\) (equivalently, \(\Delta_\theta\) may be chosen \(2\pi\)-periodic in each component).
With sparse gather/scatter, one step costs \(O(k(d_T+d_Q))\) time and \(O(k(d_T+d_Q))\) working memory, plus the cost of evaluating the small transition networks.

\paragraph{Early stopping and block output.}
Fix a maximum unroll \(T_{\max}\in\N\) and a threshold \(\varepsilon>0\). We run the transition until either \(t=T_{\max}\) or the gate becomes negligible,
\[
T(x) = \min\{t\in\{0,\ldots,T_{\max}-1\} : g_t < \varepsilon\},
\]
with the convention \(T(x)=T_{\max}\) if the set is empty. Concretely, we check \(g_t<\varepsilon\) at the beginning of step \(t\); if it holds, we stop and return \(T_t\). Otherwise, we apply the transition to produce \(T_{t+1}\) and continue.
We define the \vectur{} block output as the final tape
\[
V(x) := T_{T(x)} \in \R^{N_T\times d_T}.
\]
In downstream architectures (e.g., Llama-style models), any required reshaping or projection of \(V(x)\) is handled outside the \vectur{} block.

\paragraph{Differentiability and efficient backpropagation.}
All operations inside each step are differentiable with respect to the tape values and the transition parameters, except at the measure-zero boundaries induced by the floor/mod operations inside \(n(\theta)\). In practice, we implement reading and writing via gather/scatter on the at-most-\(2k\) active indices, which is efficient and supports backpropagation through the unrolled computation.
Early stopping introduces a discrete dependence on the stopping time \(T(x)\); a standard choice is to stop the forward pass when \(g_t<\varepsilon\) and treat the control-flow decision as non-differentiable, while gradients still flow through all executed steps (alternatively, one can always run for \(T_{\max}\) steps and rely on the multiplicative \(g_t\) factors to effectively mask later updates).

\paragraph{Concrete parameterization (used in experiments).}
We instantiate the maps \(\M_T, \M_Q, \M_I\) as linear projections, and the transition maps \(\Delta_T, \Delta_Q,\Delta_w\) as two-layer MLPs with expansion factor 4.
Specifically, we project tape symbols position-wise,
\[
\M_T(x;W) = x W_T,\quad \M_T(z;W)= z W_T,
\]
and define \(\M_Q,\M_I\) as learnable linear maps that collapse the sequence to the required shapes. Writing
\[
\mathrm{vec}(x) := [x[1];x[2];\cdots;x[N]]\in\R^{N d_x},
\]
we set
\[
\M_Q(x;W) = W_Q\,\mathrm{vec}(x)\in\R^{d_Q},\qquad
\M_I(x;W)=\big(\boldsymbol{\theta}_0,\mathbf{w}_0\big),
\]
with
\[
\boldsymbol{\theta}_0 = (W_\theta\,\mathrm{vec}(x))\bmod 2\pi \in (S^1)^k,\qquad
\mathbf{w}_0 = W_w\,\mathrm{vec}(x)\in\R^k.
\]
No constraint is imposed on \(\mathbf{w}_0\); weights may be any real numbers.

We choose \(\kappa(x;W)\) as a two-layer MLP (expansion factor 4) with a positivity constraint so that \(\kappa(x;W) > 0\).
For \(\Delta_\theta\), we parameterize periodicity by feeding \(\sin(\boldsymbol{\theta}_t)\) and \(\cos(\boldsymbol{\theta}_t)\) into an MLP; concretely,
\[
\Delta_\theta(S_t,Q_t,\boldsymbol{\theta}_t;W)
= \mathrm{MLP}_\theta\big([S_t,Q_t,\sin(\boldsymbol{\theta}_t),\cos(\boldsymbol{\theta}_t)]\big)\in\R^k.
\]

\paragraph{Algorithm (forward pass).}
Given \((T_0,Q_0,I_0)\), we iterate for \(t=0,1,\ldots,T_{\max}-1\):
\begin{enumerate}[leftmargin=*,nosep]
    \item compute \((n_{t,i},s_{t,i},n_{t,i}^+)_{i=1}^k\) from \(\boldsymbol{\theta}_t\) via the definitions above;
    \item read \(S_t\) as a \(2k\)-term weighted sum of gathered tape vectors;
    \item compute \(g_t\); if \(g_t<\varepsilon\), stop early and return \(T_t\);
    \item update \(Q_{t+1}\) and update \((\boldsymbol{\theta}_{t+1},\mathbf{w}_{t+1})\);
    \item write by scatter-adding into the at-most-\(2k\) tape locations \(\{n_{t,i},n_{t,i}^+\}_{i=1}^k\) according to \eqref{eq:vectur_tape_update};
\end{enumerate}
We return \(V(x)=T_{T(x)}\).

\section{VecTur Blocks inside Llama-style Models}
\subsection{Macro architecture}
We adopt a standard decoder-only transformer macro architecture (token embeddings, positional encoding, residual blocks, and an LM head) following Llama-family designs \citep{touvron2023llama2,dubey2024llama3}.
We then vary the \emph{block} inside each residual layer while keeping parameter count and FLOPs roughly matched.

\begin{figure}[t]
\centering
\fbox{\rule{0pt}{1.3in}\rule{0.93\linewidth}{0pt}}
\caption{\textbf{Placeholder.} Block-swap experiment: a fixed Llama-style macro architecture where the per-layer computational block is one of \{Attention, LSTM, NTM/DNC, \vectur{}, \vecstur{}\}.}
\label{fig:blockswap}
\end{figure}

\subsection{Compared blocks}
We compare the following blocks:
\begin{itemize}[leftmargin=*,nosep]
  \item \textbf{Attention block}: multi-head self-attention + SwiGLU MLP \citep{vaswani2017attention}.
  \item \textbf{LSTM block}: a gated recurrent update applied over the sequence, wrapped with residual connections \citep{hochreiter1997lstm}.
  \item \textbf{External-memory block}: an NTM/DNC-style controller with differentiable read/write heads \citep{graves2014ntm,graves2016dnc}.
  \item \textbf{\vectur{} block}: the \vectur{} transition unrolled for \(T_{\max}\) steps with learned halting \(\kappa\).
  \item \textbf{\vecstur{} block}: \vectur{} with stochastic symbols \(z\).
\end{itemize}

\section{Evaluation Benchmarks}
\subsection{Reasoning and knowledge}
We evaluate few-shot or fine-tuned performance on:
\begin{itemize}[leftmargin=*,nosep]
  \item \textbf{GSM8K} \citep{cobbe2021gsm8k} (grade-school math; exact-match accuracy),
  \item \textbf{ARC} \citep{clark2018arc} (AI2 reasoning challenge; accuracy),
  \item \textbf{HellaSwag} \citep{zellers2019hellaswag} (commonsense completion; accuracy).
\end{itemize}

\subsection{Language modeling}
We evaluate next-token prediction on \textbf{WikiText-103} \citep{merity2016wikitext} using perplexity.

\section{CompGen: Complexity-Stratified Program Generation}
\subsection{Task format}
\compgen{} consists of short Python programs \(p\) paired with inputs \(u\) and outputs \(p(u)\).
Each instance is labeled with a target complexity class \((T(n),S(n))\) in terms of input size \(n\).
Programs are generated from templates with controlled loop structure, recursion depth, and memory allocation patterns.

\begin{table}[t]
\centering
\small
\begin{tabular}{@{}lll@{}}
\toprule
\textbf{Class} & \textbf{Example family} & \textbf{Notes} \\
\midrule
\(O(n), O(1)\) & scan / reduce & single pass \\
\(O(n), O(n)\) & prefix sums & linear auxiliary array \\
\(O(n\log n), O(1)\) & sort-then-scan & comparison sorting \\
\(O(n^2), O(1)\) & nested-loop count & quadratic time \\
\(O(n^2), O(n)\) & DP table strip & quadratic time, linear space \\
\bottomrule
\end{tabular}
\caption{\textbf{Placeholder.} \compgen{} program families and intended \((T(n),S(n))\) buckets.}
\label{tab:compgen}
\end{table}

\subsection{Generalization protocol}
We train on \(n \in [n_{\min}, n_{\text{train}}]\) and evaluate on larger \(n \in (n_{\text{train}}, n_{\text{test}}]\) to measure extrapolation.
We report accuracy as a function of \(n\) and correlate effective compute (average unroll steps) with complexity class.

\section{Randomized Computation Suite}
We include tasks where access to randomness enables provable or empirical speedups:
\begin{itemize}[leftmargin=*,nosep]
  \item \textbf{Matrix product verification} (Freivalds) \citep{freivalds1977}: verify \(AB=C\) faster than multiplication.
  \item \textbf{Probabilistic primality testing} (Miller--Rabin) \citep{miller1976,rabin1980}: decide primality with bounded error.
\end{itemize}
\vecstur{} receives stochastic symbols \(z\) and learns to leverage them to reduce expected compute (as reflected by learned \(\kappa\) and early halting).

\section{Experimental Setup}
\paragraph{Model sizes.}
We instantiate models at \(\sim\)110M, 350M, and 1.3B parameters (placeholder sizes) with matched embedding width and layer count across blocks.

\paragraph{Training.}
We train on a mixture of general text (for language modeling) and \compgen{} (for compute probing), then fine-tune on downstream reasoning tasks.
We use identical optimizers, learning rate schedules, and token budgets across conditions.

\paragraph{Compute control.}
For \vectur{}/\vecstur{} we set a maximum unroll \(T_{\max}\) and learn \(\kappa(x;W)\) to modulate effective steps.
We report both task performance and measured compute (average unroll steps per token).

\section{Results (Illustrative Placeholders)}
\paragraph{Important note.}
The tables below contain \textbf{illustrative placeholder numbers} showing the intended presentation format.
Replace these with actual experimental results.

\begin{table}[t]
\centering
\small
\begin{tabular}{@{}lcccc@{}}
\toprule
\textbf{Block} & \textbf{GSM8K} & \textbf{ARC} & \textbf{HellaSwag} & \textbf{WikiText PPL} \\
\midrule
Attention & 42.1 & 54.0 & 78.3 & 18.7 \\
LSTM & 38.4 & 51.2 & 76.0 & 20.9 \\
NTM/DNC & 43.0 & 54.4 & 78.1 & 19.3 \\
\vectur{} & \textbf{46.8} & \textbf{56.1} & \textbf{79.0} & 18.9 \\
\vecstur{} & 46.2 & 55.7 & 78.9 & 18.9 \\
\bottomrule
\end{tabular}
\caption{\textbf{Placeholder.} Benchmark performance for a mid-size model at matched parameter budget. Accuracy in \%, perplexity lower is better.}
\label{tab:benchmarks}
\end{table}

\begin{figure}[t]
\centering
\fbox{\rule{0pt}{1.5in}\rule{0.93\linewidth}{0pt}}
\caption{\textbf{Placeholder.} \compgen{} extrapolation: accuracy vs.\ input size \(n\), showing \vectur{} degrades more gracefully and increases effective steps via learned \(\kappa\).}
\label{fig:compgen}
\end{figure}

\begin{table}[t]
\centering
\small
\begin{tabular}{@{}lcc@{}}
\toprule
\textbf{Model} & \textbf{Freivalds verify} & \textbf{Miller--Rabin} \\
\midrule
\vectur{} & 71.0 & 68.4 \\
\vecstur{} & \textbf{84.5} & \textbf{79.2} \\
\bottomrule
\end{tabular}
\caption{\textbf{Placeholder.} Randomized computation suite: \vecstur{} benefits from stochastic symbols \(z\).}
\label{tab:random}
\end{table}

\section{Discussion}
These illustrative results suggest \vectur{} provides a useful inductive bias for tasks requiring iterative computation and length extrapolation, while remaining compatible with modern LLM macro architectures.
\vecstur{} further improves performance on tasks where randomized strategies are advantageous.

\section{Limitations and Future Work}
This draft omits implementation details (e.g., the exact \(\mathrm{Sparse}(\cdot)\) operator, stability constraints, and efficient kernels) and uses illustrative results.
Future work should (i) benchmark on longer-context settings, (ii) analyze failure modes of learned halting \(\kappa\), and (iii) evaluate robustness across different data mixtures and training budgets.

\section*{Acknowledgments}
\emph{Placeholder.}

\begin{thebibliography}{99}

\bibitem[Turing(1936)]{turing1936}
Alan~M. Turing.
\newblock On computable numbers, with an application to the {E}ntscheidungsproblem.
\newblock \emph{Proceedings of the London Mathematical Society}, 1936.

\bibitem[Hochreiter and Schmidhuber(1997)]{hochreiter1997lstm}
Sepp Hochreiter and J{\"u}rgen Schmidhuber.
\newblock Long short-term memory.
\newblock \emph{Neural Computation}, 1997.

\bibitem[Vaswani et~al.(2017)]{vaswani2017attention}
Ashish Vaswani, Noam Shazeer, Niki Parmar, Jakob Uszkoreit, Llion Jones, Aidan~N. Gomez, {\L}ukasz Kaiser, and Illia Polosukhin.
\newblock Attention is all you need.
\newblock In \emph{NeurIPS}, 2017.

\bibitem[Graves et~al.(2014)]{graves2014ntm}
Alex Graves, Greg Wayne, and Ivo Danihelka.
\newblock Neural {T}uring machines.
\newblock \emph{arXiv:1410.5401}, 2014.

\bibitem[Graves et~al.(2016)]{graves2016dnc}
Alex Graves, Greg Wayne, Malcolm Reynolds, Tim Harley, Ivo Danihelka, Agnieszka Grabska-Barwi{\'n}ska, Sergio G{\'o}mez Colmenarejo, Edward Grefenstette, Tiago Ramalho, John Agapiou, Adri{\`a} Puigdom{\`e}nech Badia, Karl~Moritz Hermann, Yori Zwols, Georg Ostrovski, Adam Cain, Helen King, Christopher Summerfield, Phil Blunsom, Koray Kavukcuoglu, and Demis Hassabis.
\newblock Hybrid computing using a neural network with dynamic external memory.
\newblock \emph{Nature}, 2016.

\bibitem[Graves(2016)]{graves2016act}
Alex Graves.
\newblock Adaptive computation time for recurrent neural networks.
\newblock \emph{arXiv:1603.08983}, 2016.

\bibitem[Touvron et~al.(2023)]{touvron2023llama2}
Hugo Touvron, Louis Martin, Kevin Stone, Peter Albert, Amjad Almahairi, Yasmine Babaei, Nikolay Bashlykov, et~al.
\newblock {Llama 2}: Open foundation and fine-tuned chat models.
\newblock \emph{arXiv:2307.09288}, 2023.

\bibitem[Dubey et~al.(2024)]{dubey2024llama3}
Abhimanyu Dubey et~al.
\newblock The {Llama 3} herd of models.
\newblock \emph{arXiv preprint}, 2024.

\bibitem[Cobbe et~al.(2021)]{cobbe2021gsm8k}
Karl Cobbe, Vineet Kosaraju, Mohammad Bavarian, Mark Chen, Heewoo Jun, Lukas Kaiser, Matthias Plappert, et~al.
\newblock Training verifiers to solve math word problems.
\newblock \emph{arXiv:2110.14168}, 2021.

\bibitem[Clark et~al.(2018)]{clark2018arc}
Peter Clark, Isaac Cowhey, Oren Etzioni, Tushar Khot, Ashish Sabharwal, Carissa Schoenick, and Oyvind Tafjord.
\newblock Think you have solved question answering? try {ARC}, the {AI2} reasoning challenge.
\newblock \emph{arXiv:1803.05457}, 2018.

\bibitem[Zellers et~al.(2019)]{zellers2019hellaswag}
Rowan Zellers, Yonatan Bisk, Roy Schwartz, and Yejin Choi.
\newblock {HellaSwag}: Can a machine really finish your sentence?
\newblock In \emph{ACL}, 2019.

\bibitem[Merity et~al.(2016)]{merity2016wikitext}
Stephen Merity, Caiming Xiong, James Bradbury, and Richard Socher.
\newblock Pointer sentinel mixture models.
\newblock \emph{arXiv:1609.07843}, 2016.
\newblock (Introduces the WikiText-103 benchmark.)

\bibitem[Freivalds(1977)]{freivalds1977}
Raimund Freivalds.
\newblock Probabilistic machines can use less running time.
\newblock In \emph{IFIP Congress}, 1977.

\bibitem[Miller(1976)]{miller1976}
Gary~L. Miller.
\newblock Riemann's hypothesis and tests for primality.
\newblock \emph{Journal of Computer and System Sciences}, 1976.

\bibitem[Rabin(1980)]{rabin1980}
Michael~O. Rabin.
\newblock Probabilistic algorithm for testing primality.
\newblock \emph{Journal of Number Theory}, 1980.

\end{thebibliography}

\end{document}

